\chapter{実験結果と分析}
概要のみ簡単に書いています。

\section{線形モデルとしての妥当性}

\subsection{予測性能の評価}

GLRの予測性能を、回帰指標とランキング指標の両面から評価した。





\section{寄与分解の厳密性}

\subsection{4系列による予測の構造的分解}

GLRの予測は、User・Movie・Interaction・Reviewの4系列に厳密に分解される。


\subsection{情報保持率の検証}

線形モデルの重要な性質は、寄与の足し算が予測値を完全に再現できる点にある。
これを定量的に評価するため、予測値をどの程度情報損失なく再現しているかを評価した。


\subsection{説明の安定性と外的整合性}

\paragraph{説明の安定性(SHAP比較)}

GLRの寄与は重み×特徴で一意に決まるため、試行間で変動しない。
これを確認するため、背景サンプルを10回変えてKernel SHAPを再計算し、
各特徴の変動係数(CV)を比較した。

\paragraph{外的整合性(Permutation Importance)}

GLRの内部重みが外部的な重要度とも整合しているか検証するため、
Permutation Importanceによる予測誤差の増加量と比較した。




\section{個性理解の定量的実証}

本節では、GLR による寄与分解が個々の予測説明にとどまらず、
ユーザー間の評価傾向の違いを集団的な構造として捉え得るかを検証する。

\subsection{寄与バランスとVariance分析}

GLRがユーザー固有の評価傾向を捉えているかを確認するため、
評価がどの程度「人によって変わるのか」、あるいは「作品によって決まるのか」を比較した。
具体的には、同一ユーザーが異なる映画を評価した場合の寄与のばらつきと、
同一映画を異なるユーザーが評価した場合の寄与のばらつきを対比した。



\subsection{評価スタイルクラスターの発見}

GLRでは、寄与分解を通じて各ユーザーの評価スタイルを
6軸(interaction / actor / director / genre / content / background)のベクトルとして表現できる。
この6次元ベクトルをK-means(k = 6)によりクラスタリングし、評価傾向の類型化を行った。




\subsection{評価スタイル差の大きさに関する補助的検証}

前節で得られた評価スタイルの分布傾向について、
その差がどの程度の大きさを持つかを補助的に確認するため、
効果量(Cohen’s d)に基づく比較を行う。
6つのクラスター間の全15ペアについて、各軸の寄与分布を比較した。


\subsection{実例による解釈}

評価スタイルが実際の映画評価にどのように反映されるか確認するため、
各クラスタを代表する典型ユーザーの寄与構造を、
同一映画における全ユーザー平均と比較した。

