
\chapter{考察}
\label{chap:discussion}

% 本章の概要
本章では、第4章で示した実験結果を踏まえ、
本研究の意義、映画評価行動に関する示唆、推薦UIとしての位置づけ、
および本手法の限界と今後の展望について議論する。

% --------------------------------
\section{線形分解モデルとしての意義}
\label{sec:discussion_significance}

GLR は、User・Movie・Interaction・Review の 4 系列に
寄与を厳密に分解する線形構造を採用している。
この設計選択は、近年多くの推薦手法が採用する 
attention ベースの post-hoc 説明や、SHAP などの外部説明手法とは根本的に異なる。
これらは学習済みモデルに対して“近似的に”重要度を推定するアプローチであり、
説明が必ずしも予測構造と一致するとは限らない。

一方、GLR における寄与はモデルの構造そのものであり、
予測値は寄与の線形和として定義的に分解される。
このため、寄与の可加性・系列間の独立性が原理的に保証される点は、
本手法の大きな特徴である。

また、潜在因子を用いる Matrix Factorization などとは異なり、
GLR は高次元の映画特徴・ユーザー特徴をそのまま観測可能な空間で扱う。
これにより、寄与がどの特徴量から生じたかをそのまま追跡でき、
説明とモデル構造の整合性が損なわれることがない。
特徴空間を圧縮しないため、映画の俳優・監督・ジャンル・内容といった意味的な属性を
保持したまま寄与を算出できる点は、解釈性の観点から重要である。

実験結果も、線形モデルとして期待される性質と強い整合性を示した。
残差診断では線形性・等分散性・正規性がいずれも満たされ、EIL=0 という結果は、
寄与分解が情報を損なわず完全に予測値を再構成していることを示す。
これらは GLR の構造的透明性が、実験的にも矛盾なく成立していることを意味する。

さらに、GLR の正則化設計、とりわけ Movie Suppression は、
映画特徴が高次元であるがゆえに生じる Movie 過大寄与の問題を抑制する役割を果たした。
これは事前に仮定した要件ではなく、実験を重ねる中で観測された傾向に基づいて導入したものであり、
GLR が実データに即して系列構造を安定に保とうとする点でも意義がある。

以上より、GLR は単なる線形モデルではなく、高次元特徴をそのまま扱いながら、
寄与を厳密に分解し、透明性を保ったまま推薦を行うための構造的枠組みとして位置づけられる。



% --------------------------------
\section{評価スタイル可視化から得られる示唆}
\label{sec:discussion_personality}

GLR による寄与分解に基づくクラスタリングは、
ユーザーが映画を評価する際に参照する情報源の“使い方”に
一定の多様性が存在することを示した。
本研究で得られた6クラスタは、映画そのものの属性ではなく、
**寄与として観測される評価行動の傾向**に基づく分割であり、
ユーザーがどの系列の情報を相対的に多く利用しているかを反映している。
特に、Actor-Focused や Director-Focused のように
特定の系列が際立って大きいクラスタは、
寄与構造に明確な偏りが見られる点で興味深い。
これらのクラスタは、ユーザーが一貫して俳優情報・監督情報に反応しており、
その特徴が予測寄与としても安定的に表れることを示している。
このような“際立った評価スタイル”が自然に抽出されたことは、
映画評価が単一の指標ではなく、
ユーザーごとに異なる判断基準の組み合わせによって形成されている
という観点を裏付ける。

一方で、複数軸が均等に近い Balanced 型のユーザーも一定数存在する。
このクラスタは、特定の情報系列が突出しないユーザー群を
統計的にまとめたものであり、
本研究では特に取り上げて議論するものではない。
むしろ、俳優・監督・ジャンル・内容といった
**特徴的な寄与の偏りを持つクラスタが明確に現れた点**が、
本分析の意義として大きい。
また、得られたクラスタは GLR の系列分解設計と整合的であるため、
寄与の強弱は単なる統計的な変動ではなく、
“どの種類の特徴量が予測に寄与しやすいか”
というモデル内の構造と対応している。
そのため、本研究で得られた評価スタイルは、
ユーザーの認知的評価プロセスそのものを直接記述したものではないものの、
映画評価における**情報選択の傾向を形式的に捉えたもの**として解釈できる。
さらに、今回得られた6軸は、ユーザーの評価行動を把握する上で
大まかな構造を示す第一段階であり、
今後さらに細かな分析へ発展しうる基盤となる。
ただし、より詳細な個性モデルの設計や細分化の可能性については、
本章では踏み込まない。
これらはユーザースタディや非線形拡張など、
将来的な検討課題として第5.5節で述べる。

以上より、GLR を用いた評価スタイルの可視化は、
映画評価における個性の多様性を形式的に示すものであり、
俳優・監督・ジャンル・内容といった系列に基づく
**複数の情報利用パターンが安定的に存在する**ことを明らかにした。




% --------------------------------
\section{本手法の限界}
\label{sec:discussion_limitation}
本研究で構築した GLR は,寄与分解の明瞭さと系列構造の解釈可能性を重視し,
予測モデルを 完全に線形な形式に制約した 点に特徴がある。
この選択は透明性の観点では大きな利点をもたらす一方で,
表現力および柔軟性に関して固有の限界を伴う。

まず,線形モデルは入力特徴量を 加法的に組み合わせる構造 を基本としており,
評価行動にしばしば見られる 条件依存的・非線形的な好みの変化 を自動的に捉えることが難しい。
たとえば「特定の俳優 × 特定のジャンル」のような組み合わせ効果や,
高評価帯と低評価帯で活性化する基準が異なる場合,
あるいは「同じ監督でもテーマによって評価が変わる」ような文脈依存性は,
線形加法モデルでは十分に表現できない。

理論的には,交互作用項を明示的に設計すれば,
一定の条件依存構造を線形モデル内で表現することは可能である。
しかしそれは,モデルが自動的に分岐規則を学習するわけではなく,
研究者が前もって特徴量として組み込む必要がある という点で大きく異なる。
すなわち,決定木系や勾配ブースティング系のように
内部で分岐構造を生成する非線形モデルに比べると,
線形モデルは構造的に表現力が制約される。

以上より,本研究の GLR は透明性を確保する代わりに,
取り得る関数形の幅に理論的な制約が残る。
このため,寄与分解の厳密性という目的は達成されるものの,
予測性能の上限に関しては,非線形モデルと比べて一定の妥協を伴う ことは避けられない。

次に、特徴量設計そのものが抱える構造的な限界がある。

本研究の 4 系列(User / Movie / Interaction / Review)は
寄与の切り分けを明確にする上で有効であったが、
入力特徴が “人間が理解できる軸” と一対一で対応しているわけではない。
Movie 系列に含まれる埋め込み(FastText・タグベクトルなど)は、
映画間の意味的距離を高次元で近似するものであり、
その次元が具体的にどのような美学的・内容的要素を反映するのかを明確に解釈することは難しい。
同様に、User 系列は過去行動の統計要約に依存しているため、
心理的指標や認知的基準とは必ずしも一致しない。
Review 系列についても、記述量の個人差や内容の偏りにより、
評価基準が十分に反映されない場合が多い。
これらの点は、系列分解の透明性が “モデル内部の透明性” に留まり、
“特徴そのものの透明性” とは異なる性質を持つことを示している。
さらに、利用したデータセットにも限界がある。
IMDb の評価はユーザー層が偏っており、
辛口・甘口の傾向、レビューを書く習慣の有無、
映画のどの側面について書くかといった個人差が大きい。
標準化によって評価尺度の違いはある程度補正できるものの、
評価行動全体を均質化できるわけではない。
レビュー内容は実際に考慮した判断基準を必ずしも反映せず、
作品の核心的な評価ポイントが文章に現れないことも多い。
本研究で得られた個性軸は、
こうしたデータの制約の中で抽出された近似的な構造に過ぎない。
最後に、系列間の完全な分離は原理的に不可能である。
人間の映画評価は、作品固有の性質と個人の嗜好が相互作用して生まれるものであり、
「作品寄りの要因」と「個人寄りの要因」を厳密に分割することは概念的にも難しい。
Movie Suppression をはじめとする正則化により、
Movie 系列が過度に支配的になることは抑制できているものの、
両者が完全に独立した寄与として分離されているわけではない。
本手法の分離は“十分ではあるが完全ではない”という位置づけに留まる。

以上のように、GLR は透明性を優先した枠組みとして有効である一方、
その透明性を支える線形性・特徴量依存・データの偏りには、
いずれも構造的な限界が存在する。
これらの点は、次節で述べる今後の展望における重要な課題となる。


% --------------------------------
\section{今後の展望}
\label{sec:discussion_future}

本研究で示した GLR は,映画評価を系列ごとに分解し,ユーザー固有の評価スタイルを可視化するための基盤として有効である一方,その可能性は現状の枠組みにとどまらない。今後の発展としては,モデルの高度化,個性軸の深化,さらには推薦体験そのものの拡張という三つの方向から検討が考えられる。

第一に,**モデルの発展と大規模化**である。今回の実験では中規模のデータセットを用いたが,データ量が増加すれば,より高精度な寄与推定やユーザー間の細やかな差異の抽出が可能となる。特に,俳優嗜好や監督スタイルの内部構造といった,より微細な“サブ個性軸”を同定することができ,6クラスタを越えた多層的な評価パターンの抽出につながると考えられる。また,GLR の線形性は透明性を維持する大きな利点であるが,より複雑な条件依存の好みを扱うために,交互作用項の拡張や部分的な非線形化,さらには透明性を保ったまま表現力を高めるハイブリッド手法(例:Factorization Machines との統合)なども将来的な検討対象となる。透明性を失わずに表現力を拡張するという課題は,本研究の延長線上で特に重要である。

第二に,**個性軸の深化と対応関係の理解**である。本研究では評価スタイルを6軸に整理したが,これらはあくまで基底的な分類であり,より大規模なデータや高精度の特徴抽出を用いることで,俳優・監督・ジャンル・内容の内部にある多様なサブカテゴリを浮かび上がらせることが可能になる。また,映画側の特徴空間とユーザー側の個性ベクトルを対応づけるマッピングの設計により,「この作品のどの特性が,どの個性軸と強く結びついているのか」を定量的に示すことが期待される。さらに,将来的には,ユーザー自身が評価軸の重みづけを調整できるようなインタラクティブな仕組みを導入することで,**“自分の映画観を自分で編集する”** という新しい推薦の形を実現できる可能性がある。

第三に,**推薦体験そのものの拡張**が挙げられる。本研究で提示した UI は寄与分解の理解を補助するためのプロトタイプにすぎないが,GLR の特徴である透明性は,より豊かなインタフェース設計につながる潜在力を持つ。たとえば,寄与の可視化を土台として,ユーザーと対話しながら嗜好を調整していく“映画コンシェルジュ”的な対話型推薦への発展が考えられる。近年の LLM と組み合わせることで,GLR が算出する寄与を自然言語で解釈し,ユーザーの語りと結びつけながら推薦を行うような,**映画談義に近い推薦体験**を実現できると期待される。透明な寄与分解と対話型 AI の組み合わせは,従来の“隠れたアルゴリズムによる推薦”とは異なる,新たな価値を持つパーソナライズの方向性になり得る。

これらの方向性はいずれも,本研究が構築した GLR を出発点として実現可能なものであり,映画推薦における透明性と個性理解を軸とした新しい研究領域を切り開く可能性を持つ。