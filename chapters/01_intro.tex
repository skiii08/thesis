\chapter{はじめに}
\label{chap:intro}

\textbf{とりあえず書いてみたけど、研究全体を見て結局言えそうなことを最後に調整します}


---------------------------------------------------------------------

映画配信サービスの普及により、膨大な作品の中から利用者の嗜好に合った映画を提示する推薦システムは、現代の視聴体験を支える重要な役割を担っている。Netflix や Amazon Prime Video などの商用サービスでは、大規模データを用いた高精度のアルゴリズムが広く採用されており、多くのユーザーに対して効率的な作品提案を行っている。こうしたサービスは、大衆向けの設計として、シンプルで直感的な体験を優先し、推薦の仕組みをあえて前面に出さない選択をとることが一般的である。

一方で、映画推薦において「なぜこの作品なのか」、より深い納得感や選択の透明性を求めるユーザーも一定数存在する。映画の選び方は個人によって大きく異なり、ジャンルや俳優といった表層的な情報だけでなく、映像表現、音楽、テーマ性など、多様な観点から作品を評価する層も少なくない。特に、作品を独自の観点から評価するユーザーにとっては、推薦理由が見えにくい推薦は、たとえ精度が高くとも、選択の納得感を損なう場合がある。こうしたニーズは商用システムの主要ターゲットではないものの、一定の重要性を持つと考えられる。

実際、AI の判断根拠を理解しようとする動きは近年広がっており、Explainable AI(XAI)やモデル透明性に関する研究が推薦分野でも注目されている。これは、透明性のある推薦システムへの関心が、単なる個人的な興味にとどまらず、時代の要請として認識され始めていることを示している。

以上の背景から、ユーザーの評価スタイルと映画特徴の対応関係を分かりやすく示す透明な推薦手法には一定の意義がある。特に独自の嗜好を持つユーザーに対しては、その効果がより大きいと考えられる。

本研究の中心課題は、ユーザー固有の評価スタイルと映画特徴の対応関係を、安定かつ一貫した形で可視化できる推薦枠組みを構築することである。モデルの寄与を系列ごとに明確化し、どの情報がどの程度影響したかを一貫して解釈できる仕組みを整えることで、推薦結果への納得感を高めることを目指す。なお、本研究で作成した UI は、こうした寄与分解の理解を補助するプロトタイプとして位置づけ、モデルの説明可能性をどのように提示し得るかを示すための参考として扱う。

本研究は、商用推薦システムを代替するものではなく、むしろその補完として、透明性と個性理解を重視する文脈における新しい推薦体験の可能性を探るものである。